{\documentclass [10pt,a4paper]{book}
\begin{document}
\begin{flushright}
DESIGNING E-RESEARCH \textbf{33} 
\end{flushright} 

\begin{itemize}
\item   
Transformation. Transformation occurs when the use of technology becomes an end in itself. Specifically, when using a technology, we function within its structure.
\item 
Resonance. Resonance refers to the significance or enduring effect attached to the use of one technology over another. How significant the resonance is depends on the media transformations, which, in turn, derive from the nature and use of a particular technology. 
\end{itemize}

These features will filter the design of the e-research program, beginning with the selection of the research problem itself. E-researchers will likely focus on problems in which the use of the technology is integral to the innovation or problem under investigation. They will need to insure that the transparency of the components of the infrastructure and especially of the formal social system itself in which they operate are is not ignored in their focus on exiting Net-based innovations. E-researchers must also be cognizant of their often insidious motivation to act as a proponent or advocate, rather than as an observer focused on maintaining as much objectivity as possible. When we become transformed by the technology, we lose our capacity to examine it. Finally, each Net technology gains resonance in its application—supporting ur under-mining earlier modes of communication and information processing. The e-researcher is therefore challenged by these technological features to insure that he/she chronicles as accurately as possible the effect of the Net on the education or social system—rather than only its hidden effects on the individual researcher. All of these elements exist in dynamic interaction, and, as such, comprise an ecology of mediation. We encourage c-researchers to reflect on the elements that Chandler proposes throughout their e-research projects and to include them in the dissemination of findings. Doting so will help them understand the influence of the Net on the research process.
\begin{flushleft} 
\textbf{CHOOSING A RESEARCH PARADIGM} 
\end{flushleft} 
Ordinarily, educational research studies will state a purpose, pose a question, collect data on a specific population, and analyze the data, resulting in outcomes that make a contribution to the knowledge in the field. There arc, however, different ways of arriv-ing at the knowledge or methods of inquiry—most often termed quantitative research and qualitative research—including critical theory, historical and action research, liter-ature reviews, and mixed method research (Reeves, 2000). Quantitative research has been referred to as experimental or empiricist research and is associated with the pos-itivist or scientific paradigm. Qualitative research is sometimes referred to as natural-istic design, and is associated with the interpretive or constructivist approach or the post-positivist perspective. Research that uses critical theory typically involves the deconstruction of systems, the revelation of hidden agendas, and the documentation of various forms of disenfranchisement. A historical paradigm attempts to detail the objective reconstruction of historical events. A literature review involves a synthesis of existing research and often entails a meta-analysis or frequency count of reported out-comes. A mixed method, as the name implies, combines a number of methods such as 
\end{document}