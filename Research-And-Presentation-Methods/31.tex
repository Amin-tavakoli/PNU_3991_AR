{\documentclass [10pt,a4paper]{book}
\begin{document}
\begin{flushright}
DESIGNING E-RESEARCH \textbf{31} 
\end{flushright} 
assumption is itself a research question that should not be taken as a given in the state-ment of a research problem. 


The difficulty with the second question is twofold. As with the first question, it requires an absolute response. Even more problematic, however, is that it would be impossible to identify all the ways and to what populations established survey research is administered and to compare how these situations are administered on the Internet. As such, this research question is also unanswerable. Yet, with relatively minor changes in wording, both of these questions can become researchable. Examine the following questions.
\begin{flushleft}
\textbf{Example 1.} How does the use of the Internet for research influence research effectiveness?
 
\textbf{Example 2.} In what ways are online questionnaires more effective, as com-pared to paper-based questionnaires, for educational research? 
\end{flushleft}
While these questions are in need of further clarification either through subquestions (or hypotheses) that narrow the focus of the study or through definitions of key words in more precise terms (e.g., influence and effectiveness), the e-researcher can answer them, making the research doable. Good research questions are answerable and have clarity. 


Good research questions that focus on the use of the Net are also significant—sometimes referred to as the so what factor. The answer to a research question about the Net should result in advancing our understanding, significantly improving practice, or expanding our knowledge of a phenomenon. It is all too easy for beginning researchers (or even experienced researchers who are new to the Internet) to become seduced by the technology itself, rather than the affects of it, producing results that have little relevance and/or significance (or the so what factor). The research question is a most critical component in avoiding the so what factor. Many researchers write and rewrite their question numerous times. Even in the early stages of data collection (e.g., the pilot study) researchers often gain additional insight to the problem under investi-gation and further clarify and revise the research question. 


Finally, e-research questions should be structured and worded in language that is compatible with the focus of the study and the research method or paradigm. To determine the language most fitting, careful attention must be paid to the aims of the research and the validity of the selected method and tools. The research method that is most appropriate for the e-researcher will depend on the intent of the study. 


As mentioned at the beginning of this chapter, the role of the Net in the research process is multifaceted, complex, and not yet well understood. The least understood aspect of the Net is how—or in fact, whether—it influences the research process. In the 1960s, Canadian media theorist Marshall McLuhan argued that the "medium is the message." This generated a seemingly unsolvable debate over the effect of technology on communication, in particular, whether or not technologies are neu-tral and if biases arise from the ways in which technology is used or through the tech-nology itself. 

\end{document}