{\documentclass [10pt,a4paper]{book}
\begin{document}
\begin{flushleft}
\textbf{32} CHAPTER THREE
\end{flushleft} 

 
There have been convincing arguments on both sides of this debate. For exam-ple, James Kulik and associates (see Kulik, Kulik,  Cohen, 1980; Kulik, 1984; Kulik, Kulik,  Schwab, 1986) conducted a series of meta-analyses dealing with the effects of computer technologies on educational outcomes. The conclusions of these studies were that technologies either had no significant effect or a positive effect on learner achievement. Responding to claims that technologies have a positive effect, Richard Clark summarized six decades of educational media research and concluded that tech-nologies are "mere vehicles" (Clark, 1983, p. 445). Moreover, Clark observed that much of the literature that concludes that educational media has a positive effect, results from studies with deep methodological flaws that confound the way the media is used with the effect of the media itself. This now famous article by Richard Clark (in the Review of Educational Reseirmb) asserts the following: "Media are mere vebicks that deliver instruction but do not influence student achievement any more than the truck that delivers our groceries causes changes in nutrition ... only the content of the vehi-cle can influence achievement" (p. 445). The truck driver analogy that Clark uses to contextualize and bring relevance to the argument for the neutrality of technologies is held by a number of educational media researchers. In the same way, for example, Jonassen (1996) asserts that "carpenters use their tools to build things; the tools do not control the carpenter. Similarly, computers should be used as tools for helping learn-ers build knowledge; they should not control the learner" (p. 4). While both Clark's and Jonassen's arguments sound solid in their rationale, social science media theorist Chandler (1996) reminds its that the influence of media is a complex phenomenon. Building on the assumption of tht non-neutrality of technologies, Chandler postulates that media shape our experiences through their selectivity. In particular, he asserts that when we interact with media we act and are acted on, use and are used:
\begin{flushleft}
[Media] facilitates, intensifies, amplifies, enhances, or extends certain kinds of use or experience whilst inhibiting, restricting or reducing other kinds ... there are losses as well as gains. A medium closes some doors as well as opening others, excludes as well as includes, distorts as well as clarifies, conceals as well as reveals, denies as well as affirms, destroys as well as creates. The selectivity of media tends to suggest that some aspects of experience are important or relevant and that others are unimportant or irrelevant. Particular realties are thus made more or less accessible—more or less .rear—by different processes of mediation.
\end{flushleft}
Chandler has identified the key features of technological impact that reshape many of our social structures as:

\begin{itemize}
\item  
Selectivio, Selectivity is the way a technology is formalized within its own system. The outcome of selectivity occurs when some aspects of our experiences are amplified through the use of communication technologies causing others, in turn, to be reduced. 
\item 
Transparency occurs when we become so familiar with a technol-ogy that we do not recognize its influence and hence, the consequences of its usc. This results in an inability to exercise our choices.

\end{itemize} 
\end{document}