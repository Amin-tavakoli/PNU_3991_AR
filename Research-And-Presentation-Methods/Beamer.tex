\documentclass [12pt]{beamer}
\usepackage{xcolor}	
\usepackage{tikz}
\usetheme{AnnArbor}
\usetheme{Warsaw}
\usepackage{ragged2e}
\begin{document}

\section*{kholase safahat 31...33}
\subsection*{Amin tavakoli}	
\begin{frame}

\justifying
	
While these questions are in need of further clarification either through sub questions that narrow the focus of the study or through definitions of key words in more precise terms the e-researcher can answer them. The answer to a research questions focused on the Net should result in advancing our understanding, significantly improve practice, or expanding our knowledge of a fact. Finally, e-research questions should be structured and worded in language that is compatible with the fucus of the study and the research method or paradigm. To determine the language most fitting careful attention must be paid to the aims of the research and validity of the selected method and tools.

\end{frame}

\begin{frame}

\justifying
	
The conclusions of Kulik studies were that technology ethier had no significant effect or a positive effect on learner achievement and clark concluded that technologies are mere vehicles. While Clarks arguments sounds solid in their rationale, Chandler reminds that the influence of media is a complex fact, Chandler suppose that media shape our experiences through their selectivity. Media amplifies certain kinds of experiences while reducing other kinds. Key features of technology impacts:	1) Selectivity is the way a technology is formalized within its own system. The outcome of it occurs when some aspects of experiences are amplified through the use of technology causing others to be reduced.

\end{frame}

\begin{frame}

\justifying
	
2)Transparency occurs when we become so familiar with the technology that we don not recognize its influence and hence	3)Transformation occurs when the use of technology becomes an end itself	4)Resonance refers to significant or enduring effect attached to the use of technology over another.how significant the resonance is depends on transformation Choosing a research paradigm: there are different ways of arriving at the knowledge or mehode of inquiry most often termed quantitative research and qualitative research. Quantitative research has been referred to as experimental or empiristic research and is associated with the scientific paradigm. Qualitative research is sometimes referred to as naturalistic design and is associated with the interpretive or constructivist approach.

\end{frame}
\end{document}	

